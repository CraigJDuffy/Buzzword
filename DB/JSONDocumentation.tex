\documentclass[12pt, a4paper]{article}

\usepackage[utf8]{inputenc}

\usepackage{minted} % Look at the pretty colours, and indenting, and wrapping...
\AtBeginEnvironment{minted}{
  \renewcommand{\fcolorbox}[4][]{#4}} % prevents red error box appearing for parsing errors

\begin{document}
\title{Component Intercommunication JSON Specification}
\author{Tommy Lamb}

\maketitle

\tableofcontents
\pagebreak

\section{Menu}

\begin{itemize}
\item The Menu object primarily consists of a tree of two other objects: Menu Items and Menu Sections %TODO link

\item MenuID is the globally unique identifier for this menu object

\item DisplayName is the string to be used if representing the object to a user

\item Sections is a list of the top-level MenuSection objects that comprises this menu object. Examples may include MenuSection objects representing Starters, Mains, Deserts, or Drinks.
\end{itemize}

\begin{minted}{json}
Menu : {
	"MenuID" : int,
	"DisplayName" : string,
	"Sections" : [MenuSection, MenuSection, ...]
}
\end{minted}


\subsection{Menu Item}

\begin{itemize}
\item Represents any single 'thing' that can be ordered by a customer.
\item The ItemID is a globablly unique identifier for the specific item.
\item ParentSectionID is the globally unique identifier for the menu section under which this item is listed.
\item DisplayName is the string to be used if representing the object to a user
\item Description is a textual description of the object, eg \textit{Decadent chocolate Bombe glacée drizzled with toffee sauce and served with black forest fruit compote.}
\item Price is the decimal value representing the price of the item to the customer ordering.
\end{itemize}

\begin{minted}{json}
MenuItem : {
	"ItemID" : int,
	"ParentSectionID" : int,
	"DisplayName" : string,
	"Description" : string,
	"Price" : decimal
}
\end{minted}


\subsection{Menu Section}

\begin{itemize}
\item Represents a section of the menu, which may contain either a number of subsections, or a number of menu items.
\item ParentSectionID is the globally unique identifier for the menu section under which this section is listed.
\item ParentSectionID may be null if the section is a top-level element. That is, child only to the Menu object.
\item GroupColour is a string containing a colour hex code, eg "\#FFFFFF"
\item HasItems represents whether the section has items (true) or subsections (false).
\item Behaviour where both Items and Subsections are not null is as yet undefined. 
\end{itemize}

\begin{minted}{json}
MenuSection : {
	"SectionID" : int,
	"ParentSectionID" : int,
	"DisplayName" : string,
	"GroupColour" : string,
	"HasItems" : Boolean,
	"Items" : [MenuItem, MenuItem, ...],
	"Subsections" : [MenuSection, MenuSection, ...]
}
\end{minted}



\section{Order}

\begin{itemize}
\item Represents an order as placed by a customer (by any means).
\item OrderNumber is a number that can be used to geographically locate the customer and/or to group orders entered separately. For example, a table number.\linebreak
For grouping separate orders, this allows orders to be placed separately from a group of people, but have those orders conceptually grouped in the system to allow better customer service (eg have all orders prepared at the same time).
\item Items is the list of OrderItem objects that this order consists of.
\item ETA is either the string HH:MM representing the estimated time of completion for the order, or null if no such estimate exists.
\end{itemize}

\begin{minted}{json}
Order : {
	"OrderNumber" : int,
	"OrderName" : string,
	"ETA" :  string | null,
	"Items" : [OrderItem, OrderItem, ...],
}
\end{minted}

\subsection{Order Item}

\begin{itemize}
\item Represents a single type of MenuItem ordered by a customer.
\item Amount is the non-zero integer representing how many of the MenuItem the customer has ordered.
\item Request is a potentially empty string which represents any specific requests or choices made by the customer (as under F-UR-1.4 and related Non-Functional requirements)
\item MenuItem is the JSON object specified above representing the thing being ordered.
\item ETA is either the string HH:MM representing the estimated time of completion for this item, or null if no such estimate exists.
\item The Modified flag must be specified if the Order object is being used in the context of updating the details of an order.
\end{itemize}

\begin{minted}{json}
OrderItem : {
	"Amount" : int,
	"Request" : string,
	"MenuItem" : MenuItem,
	"ETA" : string | null,
	["Modified" : Bool]
}
\end{minted}
\end{document}